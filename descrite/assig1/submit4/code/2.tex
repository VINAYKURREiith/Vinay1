\let\negmedspace\undefined
\let\negthickspace\undefined
\documentclass[a4,12pt,onecolumn]{IEEEtran}
\usepackage{amsmath,amssymb,amsfonts,amsthm}
\usepackage{algorithmic}
\usepackage{graphicx}
\usepackage{textcomp}
\usepackage{xcolor}
\usepackage{txfonts}
\usepackage{listings}
\usepackage{enumitem}
\usepackage{mathtools}
\usepackage{gensymb}
\usepackage[breaklinks=true]{hyperref}
\usepackage{tkz-euclide}
\usepackage{listings}
\DeclareMathOperator*{\Res}{Res}
\renewcommand\thesection{\arabic{section}}
\renewcommand\thesubsection{\thesection.\arabic{subsection}}
\renewcommand\thesubsubsection{\thesubsection.\arabic{subsubsection}}
\renewcommand\thesectiondis{\arabic{section}}
\renewcommand\thesubsectiondis{\thesectiondis.\arabic{subsection}}
\renewcommand\thesubsubsectiondis{\thesubsectiondis.\arabic{subsubsection}}
\hyphenation{op-tical net-works semi-conduc-tor}
\def\inputGnumericTable{}                                
\lstset{
frame=single, 
breaklines=true,
columns=fullflexible
}
\begin{document}
\newtheorem{theorem}{Theorem}[section]
\newtheorem{problem}{Problem}
\newtheorem{proposition}{Proposition}[section]
\newtheorem{lemma}{Lemma}[section]
\newtheorem{corollary}[theorem]{Corollary}
\newtheorem{example}{Example}[section]
\newtheorem{definition}[problem]{Definition}
\newcommand{\BEQA}{\begin{eqnarray}}
\newcommand{\EEQA}{\end{eqnarray}}
\newcommand{\define}{\stackrel{\triangle}{=}}
\bibliographystyle{IEEEtran}
\providecommand{\mbf}{\mathbf}
\providecommand{\pr}[1]{\ensuremath{\Pr\left(#1\right)}}
\providecommand{\qfunc}[1]{\ensuremath{Q\left(#1\right)}}
\providecommand{\sbrak}[1]{\ensuremath{{}\left[#1\right]}}
\providecommand{\lsbrak}[1]{\ensuremath{{}\left[#1\right.}}
\providecommand{\rsbrak}[1]{\ensuremath{{}\left.#1\right]}}
\providecommand{\brak}[1]{\ensuremath{\left(#1\right)}}
\providecommand{\lbrak}[1]{\ensuremath{\left(#1\right.}}
\providecommand{\rbrak}[1]{\ensuremath{\left.#1\right)}}
\providecommand{\cbrak}[1]{\ensuremath{\left\{#1\right\}}}
\providecommand{\lcbrak}[1]{\ensuremath{\left\{#1\right.}}
\providecommand{\rcbrak}[1]{\ensuremath{\left.#1\right\}}}
\theoremstyle{remark}
\newtheorem{rem}{Remark}
\newcommand{\sgn}{\mathop{\mathrm{sgn}}}
\providecommand{\res}[1]{\Res\displaylimits_{#1}} 
\providecommand{\mtx}[1]{\mathbf{#1}}
\providecommand{\fourier}{\overset{\mathcal{F}}{ \rightleftharpoons}}
\providecommand{\system}{\overset{\mathcal{H}}{ \longleftrightarrow}}
\newcommand{\solution}{\noindent \textbf{Solution: }}
\newcommand{\cosec}{\,\text{cosec}\,}
\providecommand{\dec}[2]{\ensuremath{\overset{#1}{\underset{#2}{\gtrless}}}}
\newcommand{\myvec}[1]{\ensuremath{\begin{pmatrix}#1\end{pmatrix}}}
\newcommand{\mydet}[1]{\ensuremath{\begin{vmatrix}#1\end{vmatrix}}}
\let\vec\mathbf
\title{
\Huge\textbf{Discrete Assignment}\\
\Huge\textbf{EE1205} Signals and Systems\\
}
\large\author{Kurre Vinay\\EE23BTECH11036}
\maketitle
\bigskip
\renewcommand{\thefigure}{\theenumi}
\renewcommand{\thetable}{\theenumi}
\textbf{Question 11.9.3.8:}
Find the sum to indicated number of term in each of the geometric progressions in $\sqrt{7} ,\sqrt{21} , 3\sqrt{7}, ....n$ terms\\
\textbf{Solution:}
 Sum of the geometric progression of $\sqrt{7}, \sqrt{21}, 3\sqrt{7},....n$ terms is\\
 \textbf{Input Table:}
 \begin{center}
\begin{tabular}{|c|c|c|}
   \hline
   variable&value&description  \\
   \hline
   x(0) & $ \sqrt{7} $& first term of the geometric progession\\
   \hline
   r & $\sqrt{3}$ & common ratio of the geometeric progression\\
   \hline
   x(n) & $\sqrt{7*3^{(n)}}$& $n^{th}$ term of the geometric progession\\
   \hline
   n& &no of the term in the geometric progression\\
   \hline
   y(n+1) &$\frac{x(0)(r^{n+1}-1)}{r-1}$ &Sum of the n+1 term of the geometric progression\\
   \hline 
   U(z)&$\frac{1}{1-z^{-1}} \quad{|z^{-1}|<1}$&z-transformation of u(n)\\
   \hline
\end{tabular}
\end{center}
\textbf{\large{Z-Transformation:}}
 \begin{align}
  x(n) &= x(0)r^{(n)} \\
 X(z)\system{Z}\{x(n)\} &= \sum_{n=-\infty}^{\infty} x(n)z^{-n}\\
 &=\sum_{n=0}^{\infty} x(n)z^{-n} \\
&=\sum_{n=0}^{\infty} x(0)r^nz^{-n} \\ 
&=x(0)\sum_{n=0}^{\infty} r^nz^{-n} \\ 
&=x(0)(z^0r^0U(z)+ r^1z^{-1}U(z)+r^2z^{-2}U(z)+r^3z^{-3}U(z)+r^4z^{-4}U(z)+.......)\\
&= x(0)(1+r^1z^{-1}+r^2z^{-2}+r^3z^{-3}+r^4z^{-4}+r^5z^{-5}+r^6z^{-6}+.......)\\
X(Z)&=x(0)\brak{\frac{1}{1-rz^{-1}}},\quad{|rz^{-1}|<1}\\
y(n)&=x(n)u(n)\\
Y(z)&=X(z)U(z)\\
&=x(0)\brak{\frac{1}{1-rz^{-1}}}\brak{\frac{1}{1-z^{-1}}}\\
\textbf{\large{Contour Integration:}}\\
y(n+1)&=\frac{1}{2\pi{}j}\oint_C {Y(Z)z^{n}}dz\\
&=\frac{1}{2\pi{}j}\oint_C {x(0)\brak{\frac{1}{1-rz^{-1}}}\brak{\frac{1}{1-z^{-1}}}z^{n}}dz\\
&=\frac{1}{2\pi{}j}\oint_C {\sqrt{7}\brak{\frac{1}{1-\sqrt{3}z^{-1}}}\brak{\frac{1}{1-z^{-1}}}z^{n}}dz\\
&=\frac{\sqrt{7}}{2\pi{}j}\oint_C {\brak{\frac{1}{1-\sqrt{3}z^{-1}}}\brak{\frac{1}{1-z^{-1}}}z^{n}}dz,\quad{|z|>$\sqrt{3}$}\\
\end{align}
\end{document}
