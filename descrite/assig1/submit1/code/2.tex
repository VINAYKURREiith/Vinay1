\let\negmedspace\undefined
\let\negthickspace\undefined
\documentclass[a4,12pt,onecolumn]{IEEEtran}
\usepackage{amsmath,amssymb,amsfonts,amsthm}
\usepackage{algorithmic}
\usepackage{graphicx}
\usepackage{textcomp}
\usepackage{xcolor}
\usepackage{txfonts}
\usepackage{listings}
\usepackage{enumitem}
\usepackage{mathtools}
\usepackage{gensymb}
\usepackage[breaklinks=true]{hyperref}
\usepackage{tkz-euclide}
\usepackage{listings}
\DeclareMathOperator*{\Res}{Res}
\renewcommand\thesection{\arabic{section}}
\renewcommand\thesubsection{\thesection.\arabic{subsection}}
\renewcommand\thesubsubsection{\thesubsection.\arabic{subsubsection}}
\renewcommand\thesectiondis{\arabic{section}}
\renewcommand\thesubsectiondis{\thesectiondis.\arabic{subsection}}
\renewcommand\thesubsubsectiondis{\thesubsectiondis.\arabic{subsubsection}}
\hyphenation{op-tical net-works semi-conduc-tor}
\def\inputGnumericTable{}                                
\lstset{
frame=single, 
breaklines=true,
columns=fullflexible
}
\begin{document}
\newtheorem{theorem}{Theorem}[section]
\newtheorem{problem}{Problem}
\newtheorem{proposition}{Proposition}[section]
\newtheorem{lemma}{Lemma}[section]
\newtheorem{corollary}[theorem]{Corollary}
\newtheorem{example}{Example}[section]
\newtheorem{definition}[problem]{Definition}
\newcommand{\BEQA}{\begin{eqnarray}}
\newcommand{\EEQA}{\end{eqnarray}}
\newcommand{\define}{\stackrel{\triangle}{=}}
\bibliographystyle{IEEEtran}
\providecommand{\mbf}{\mathbf}
\providecommand{\pr}[1]{\ensuremath{\Pr\left(#1\right)}}
\providecommand{\qfunc}[1]{\ensuremath{Q\left(#1\right)}}
\providecommand{\sbrak}[1]{\ensuremath{{}\left[#1\right]}}
\providecommand{\lsbrak}[1]{\ensuremath{{}\left[#1\right.}}
\providecommand{\rsbrak}[1]{\ensuremath{{}\left.#1\right]}}
\providecommand{\brak}[1]{\ensuremath{\left(#1\right)}}
\providecommand{\lbrak}[1]{\ensuremath{\left(#1\right.}}
\providecommand{\rbrak}[1]{\ensuremath{\left.#1\right)}}
\providecommand{\cbrak}[1]{\ensuremath{\left\{#1\right\}}}
\providecommand{\lcbrak}[1]{\ensuremath{\left\{#1\right.}}
\providecommand{\rcbrak}[1]{\ensuremath{\left.#1\right\}}}
\theoremstyle{remark}
\newtheorem{rem}{Remark}
\newcommand{\sgn}{\mathop{\mathrm{sgn}}}
\providecommand{\res}[1]{\Res\displaylimits_{#1}} 
\providecommand{\mtx}[1]{\mathbf{#1}}
\providecommand{\fourier}{\overset{\mathcal{F}}{ \rightleftharpoons}}
\providecommand{\system}{\overset{\mathcal{H}}{ \longleftrightarrow}}
\newcommand{\solution}{\noindent \textbf{Solution: }}
\newcommand{\cosec}{\,\text{cosec}\,}
\providecommand{\dec}[2]{\ensuremath{\overset{#1}{\underset{#2}{\gtrless}}}}
\newcommand{\myvec}[1]{\ensuremath{\begin{pmatrix}#1\end{pmatrix}}}
\newcommand{\mydet}[1]{\ensuremath{\begin{vmatrix}#1\end{vmatrix}}}
\let\vec\mathbf
\title{
\Huge\textbf{Discrete Assignment}\\
\Huge\textbf{EE1205} Signals and Systems\\
}
\large\author{Kurre Vinay\\EE23BTECH11036}
\maketitle
\bigskip
\renewcommand{\thefigure}{\theenumi}
\renewcommand{\thetable}{\theenumi}
\textbf{Question 11.9.3.8:}
Find the sum to indicated number of term in each of the geometric progressions in $\sqrt{7} ,\sqrt{21} , 3\sqrt{7}, ....n$ terms\\
\textbf{Solution:}
 Sum of the geometric progression of $\sqrt{7}, \sqrt{21}, 3\sqrt{7},....n$ terms is\\
  The common ratio of geometric progression  is\begin{align}r &= \frac{a_2}{a_1}\end{align}\\
   common ratio\begin{align} r &= \frac{\sqrt{21}}{\sqrt{7}}\end{align}
   \begin{align} &= \sqrt{3} \end{align}\\
   first term of the geometric progression is \begin{align}  x(0) &= \sqrt{7}\end{align}\\ 
      x(n) is the $n^{th}$ term of the geometric progression\begin{align}  x(n) &= x(0)*r^{(n)}\end{align}
    \begin{align}  x(n) &= x(0)*\sqrt{3^{(n)}}\end{align}
    \begin{align}  x(n) &= \sqrt{7*3^{(n)}}\end{align}\\ 
   sum of n term in geometric progression is \begin{align} S_n &=\frac{x(0)(r^n)}{r-1}\end{align}\\
Then,
 Sum of n term of given geometric progression is \begin{align} S_n &= \frac{\sqrt{7}(\sqrt{3}^n)}{(\sqrt{3}-1)}\end{align}
 \begin{align}$\sqrt{7}+\sqrt{21}+3\sqrt{7}+...n$ terms &= \frac{\sqrt{7}(\sqrt{3}^n)}{(\sqrt{3}-1)}
\end{align}
\large{Z-Transformation:}
 \begin{align}
  x(n) &= x(0)*r^{(n)} \end{align} 
\begin{align} X(z) = \mathcal{Z}\{x(n)\} = \sum_{n=-\infty}^{\infty} x(n)z^{-n}\end{align} 
 \begin{align}&=\sum_{n=-\infty}^{\infty} x(n)z^{-n} \end{align} 
\begin{align}&=\sum_{n=-\infty}^{\infty} x(0)*r^n*z^{-n} \end{align} 
\begin{align}&=x(0)\sum_{n=-\infty}^{\infty} r^nz^{-n} \end{align} 
\begin{align}&= x(0)(1+r^1*z^{-1}+r^2*z^{-2}+r^3*z^{-3}+r^4*z^{-4}+r^5*z^{-5}+r^6*z^{-6}+.......)\end{align} 
\begin{align}X(Z)&=x(0)*\frac{1}{1-r*z^{-1}}\hspace{4cm}where r*z^{-1}<1\end{align} 
\textbf{Input Table:}\\
\\
\begin{center}
\begin{tabular}{|c|c|c|}
   \hline
   variable&value&description  \\
   \hline
   x(0) & $ \sqrt{7} $& first term of the geometric progession\\
   \hline
   r & $\sqrt{3}$ & common ratio of the geometeric progression\\
   \hline
   x(n) & $\sqrt{7*3^{(n)}}$& $n^{th}$ term of the geometric progession\\
   \hline
   n& &no of the term in the geometric progression\\
   \hline
  
\end{tabular}
\end{center}
\end{document}
