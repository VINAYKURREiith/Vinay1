\let\negmedspace\undefined
\let\negthickspace\undefined
\documentclass[a4,12pt,onecolumn]{IEEEtran}
\usepackage{amsmath,amssymb,amsfonts,amsthm}
\usepackage{algorithmic}
\usepackage{graphicx}
\usepackage{textcomp}
\usepackage{xcolor}
\usepackage{txfonts}
\usepackage{listings}
\usepackage{enumitem}
\usepackage{mathtools}
\usepackage{gensymb}
\usepackage[breaklinks=true]{hyperref}
\usepackage{tkz-euclide}
\usepackage{listings}
\usepackage{circuitikz}
\usepackage{gvv}
\begin{document}
\title{
\Huge\textbf{ GATE 2022 Assignment}\\
\Huge\textbf{EE1205} Signals and Systems\\
}
\large\author{Kurre Vinay\\EE23BTECH11036}
\maketitle
\textbf{Question:}\\
Let $x_1\brak{t} = e^{-t}u\brak{t}$ and $x_2\brak{t} =u\brak{t}-u\brak{t-2}$, where $u\brak{.}$ denotes the unit step function. If $y\brak{t}$ denotes the convolution of $x_1\brak{t}$ and $x_2\brak{t}$ ,then $\lim\limits_{t \to \infty} y\brak{t}$ = \underline{\hspace{1cm}}. (Rounded off to one decimal place)\\
\hfill(GATE EC 2022 )\\
\solution\\
\begin{table}[ht!]
\begin{center}
\label{table1:example}
\begin{tabular}{|c|c|c|}
   \hline
   variable&value&description\\
   \hline
   $x_1\brak{t}$& $e^{-t}u\brak{t}$&given function 1\\
   \hline
   $x_2\brak{t}$&$u\brak{t}-u\brak{t-2}$&given function 2\\
    \hline
    $y\brak{t}$&-& convolution of $x_1\brak{t}$ and $x_2\brak{t}$\\
    \hline
\end{tabular}
\caption{Table: Input Parameters}
\label{tab:1}
\end{center}
\end{table}
\begin{align}
y\brak{t} &= x_1\brak{t}*x_2\brak{t}
\end{align}
from \tabref{tab:1}
\begin{align}
y\brak{t} &= e^{-t}u\brak{t}*\brak{u\brak{t}-u\brak{t-2}}
\end{align}
By applying Laplace transform
\begin{align}
Y\brak{s}&=X_1\brak{s}.X_2\brak{s}\\
e^{-t}u\brak{t}&\xleftrightarrow{\mathcal{L}}\frac{1}{1+s} , \quad{ \text{Re}\brak{s}>-1}\\
u\brak{t}-u\brak{t-2}&\xleftrightarrow{\mathcal{L}}\frac{1-e^{-2s}}{s} ,\quad{\text{Re}\brak{s} > 0}\\
Y\brak{s}&=\brak{\frac{1}{1+s}}\brak{\frac{1-e^{-2s}}{s}},\quad{\text{Re}\brak{s} > 0}\\
&=\frac{1-e^{-2s}}{s\brak{s+1}}
\end{align}
 Final value theorem

\begin{align}
\lim\limits_{t \to \infty} x\brak{t} &= \lim\limits_{s \to 0} sX\brak{s}\\
\end{align}
\text{Proof:}
\begin{align}
\mathcal{L}\left[x\brak{t}\right] = &X\brak{s} = \int_{0}^{\infty} x\brak{t}e^{-st} \, dt \\
\mathcal{L}\left[\frac{dx\brak{t}}{dt}\right] &= \int_{0}^{\infty}\frac{d}{dt}\brak{x\brak{t}e^{-st}} \, dt \\
&= sX\brak{s} - x\brak{0^-}\\
 \lim\limits_{s \to 0}\left[ \int_{0}^{\infty}\frac{d}{dt}\brak{x\brak{t}e^{-st}} \, dt\right]&=\lim\limits_{s \to 0} \left[sX\brak{s} - x\brak{0^-}\right]\\
 \implies \int_{0}^{\infty}\frac{dx\brak{t}}{dt} \, dt &= \lim\limits_{s \to 0}\left[sX\brak{s} - x\brak{0^-}\right]\\
 \implies \left[x\brak{t} \right]_0^{\infty} &=\lim\limits_{s \to 0} \left[sX\brak{s} - x\brak{0^-}\right]\\
 \implies x\brak{\infty}-x\brak{0^-}&=\lim\limits_{s \to 0} \left[sX\brak{s} - x\brak{0^-}\right]\\
 \implies x\brak{\infty}&=\lim\limits_{s \to 0} sX\brak{s}\\
 \lim\limits_{t \to \infty} x\brak{t} = & x\brak{\infty} = \lim\limits_{s \to 0} sX\brak{s}
\end{align}
By applying Final value theorem
\begin{align}
\lim\limits_{t \to \infty} y\brak{t}&= \lim\limits_{s \to 0} sY\brak{s}\\
&=\lim\limits_{s \to 0} s\brak{\frac{1-e^{-2s}}{s\brak{s+1}}}\\
&=\lim\limits_{s \to 0} \brak{\frac{1-e^{-2s}}{\brak{s+1}}}\\
&= \brak{\frac{1-e^{0}}{0+1}}\\
\lim\limits_{t \to \infty} y\brak{t} &= 0
\end{align}

\end{document}
